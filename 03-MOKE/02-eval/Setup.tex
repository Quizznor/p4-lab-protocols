%!TEX root = ../main.tex

\section{Experimental setup}
\label{sec:setup}

\begin{figure}
	\label{fig:setup}
	\includegraphics[width=1.0\textwidth]{./fig/setup.png}
	\caption{}{Experimental setup of the MOKE lab, image adopted with changes from \cite{lab-manual}}
\end{figure}

As can be seen in \autoref{fig:setup}, a sample can be placed inside of an
electromagnet with controllable $B$-field. The measurement apparatus that will test
the magneto-optic Kerr effect consists of a laser diode that shines polarised light
onto the sample. Two polarisation filters and focusing lenses exist to tweak the
properties of light beam and improve the overall signal strength. The light that gets
reflected off the magnetic sample travels trough a beam splitter into two photo
diodes that in turn feed a differential amplifier. A Hall probe monitors the strength
of the $B$-field in which the sample is placed.
