%!TEX root = ../main.tex

\section{Ferrimagnets: Fe$_{1-x}$Gd$_x$ samples}
\label{sec:ferrimagnets}

Lastly, layered systems of Iron (Fe) and Gadolinium (Gd) are tested for their
magnetic properties. Several samples with varying Gd-concentrations of $x\,\%$ are
available to be experimented on. Unfortunately only one of the samples ($\#6$)
delivered physically sensible data at the time of the experiment. As a consequence,
only the set of measurements pertaining to this sample will be analysed in depth.

During measurements, the sample is placed on a heating module, whose temperature can
be controlled by adjusting an applied voltage. Due to large fluctuations in
temperature during measurements, the uneven contact of the sample to the heating
element as well as other environmental factors, the temperature of the sample is
assumed to have an uncertainty of $\pm\SI{1}{\celsius}$. After the hysteresis at a
specific temperature is measured, the temperature of the sample is changed and the
process is repeated for a different temperature. The results of this survey are
fitted to the theoretical model introduced in \autoref{sec:ferromagnet}. The gathered
model parameters are presented in \autoref{tab:fitvals-ferri}. A select few
measurement are also visualised in \autoref{fig:ferrimagnet-measurement}. As can be
seen in both the table and the figure, the coercive field strength ($H_c$) decreases
with increasing temperature. This is expected. The two antiparallel magnetic
sublattices in the sample react differently to temperature. As a consequence, the net
magnetisation of the sample varies throughout the measurement process. At a specific
temperature $T_\text{comp.}$, the magnetisation of the lattices are exactly opposite,
causing a net magnetisation of zero. This is best seen by plotting the coercive field
strength $H_c$ over sample temperature $T$, as is done in
\autoref{fig:compensation-temperature}. Fitting a third order polynomial to the data
points reveals, that the coercive field in this remperature regime can be roughly
described by \autoref{eq:compensation-polynomial}.

\begin{equation}
\label{eq:compensation-polynomial}
	H_c(T) = -0.24(2)\,\SI{}{\milli\oersted\per\celsius\cubed} \;T^3 + 0.042(3)\,\SI{}{\oersted\per\celsius\squared} \;T^2 + -3.0(1)\,\SI{}{\oersted\per\celsius} \;T + 94.6(8)\,\SI{}{\oersted}
\end{equation}

The compensation temperature is then given as the temperature, where the coercive
field is zero. Uncertainties in this result can be approximated using a maximum
error estimation with the values given in \autoref{eq:compensation-polynomial}.

\begin{equation*}
	T_\text{comp.} = 81.4^{+1.0}_{-0.9}\,\SI{}{\celsius}
\end{equation*}

Corollarily, the Gd concentration $x$ can be estimated using the formula given in
\cite{lab-manual}.

\begin{align*}
\label{eq:Gd-concentration}
	x\,(\text{at} \%) &= 0.01783\;\;T_\text{comp.} + 21 \\[7pt]
	&= 22.451^{+0.017}_{-0.016}\,\%
\end{align*}

The influence of compensation temperature on the Kerr-angle at saturation $\Upphi_s$
remains the last unanswered question. For low temperatures, the magnetisation of the
gadolinium dominates the magnetic properties of the sample, but because Gd has a
low Curie-temperature than iron, it is expected that for higher temperatures the
magnetisation of the iron sublattice becomes relevant. Accordingly, the Kerr-angle
at saturation $\Upphi_s$ is expected to drop with increasing compensation temperature.
This can however not be verified experimentally.

\begin{figure}
	\centering
	\includegraphics[width=1.0\textwidth]{./fig/ferrimagnet_measurement.png}
	\caption{The hysteresis of a FeGd layer system for different temperatures}
	\label{fig:ferrimagnet-measurement}
\end{figure}



\begingroup
\renewcommand{\arraystretch}{1.1}
\begin{table}
	\begin{center}
	\caption{Hysteresis best fit values for different magnet temperatures. Statistical errors estimated by the fitting algorithm are again neglected due to their small relative sizes.}
	\begin{tabular*}{\textwidth}{@{\extracolsep{\fill}} c|cccc}
  \toprule
	\hline
  $d$ & $\Upphi_s$ & $A$ & $H_c$ & $\mu$ \\
	\hline
  \SI{00}{\degree\celsius} & \SI{58.6761}{\milli\degree & \SI{149.7627}{\per\milli\oersted} & \SI{94.93}{\oersted} & \SI{26.36}{\micro\degree\per\oersted} \\
  \SI{10}{\degree\celsius} & \SI{55.2614}{\milli\degree & \SI{123.4514}{\per\milli\oersted} & \SI{69.87}{\oersted} & \SI{25.02}{\micro\degree\per\oersted} \\
  \SI{20}{\degree\celsius} & \SI{53.0343}{\milli\degree & \SI{68.1633}{\per\milli\oersted} & \SI{50.90}{\oersted} & \SI{26.50}{\micro\degree\per\oersted} \\
  \SI{30}{\degree\celsius} & \SI{50.1260}{\milli\degree & \SI{45.6059}{\per\milli\oersted} & \SI{35.93}{\oersted} & \SI{28.72}{\micro\degree\per\oersted} \\
  \SI{40}{\degree\celsius} & \SI{47.5209}{\milli\degree & \SI{26.9361}{\per\milli\oersted} & \SI{25.27}{\oersted} & \SI{30.24}{\micro\degree\per\oersted} \\
  \SI{50}{\degree\celsius} & \SI{45.1964}{\milli\degree & \SI{18.8061}{\per\milli\oersted} & \SI{19.83}{\oersted} & \SI{32.16}{\micro\degree\per\oersted} \\
  \SI{60}{\degree\celsius} & \SI{41.9797}{\milli\degree & \SI{12.2294}{\per\milli\oersted} & \SI{14.17}{\oersted} & \SI{34.46}{\micro\degree\per\oersted} \\
  \SI{70}{\degree\celsius} & \SI{38.8206}{\milli\degree & \SI{10.0000}{\per\milli\oersted} & \SI{9.60}{\oersted} & \SI{36.90}{\micro\degree\per\oersted} \\
  \SI{72}{\degree\celsius} & \SI{37.3744}{\milli\degree & \SI{10.0000}{\per\milli\oersted} & \SI{7.47}{\oersted} & \SI{37.96}{\micro\degree\per\oersted} \\
  \SI{75}{\degree\celsius} & \SI{37.2512}{\milli\degree & \SI{10.0000}{\per\milli\oersted} & \SI{6.18}{\oersted} & \SI{38.06}{\micro\degree\per\oersted} \\
  \SI{80}{\degree\celsius} & \SI{33.3113}{\milli\degree & \SI{10.0000}{\per\milli\oersted} & \SI{-0.51}{\oersted} & \SI{40.87}{\micro\degree\per\oersted} \\
	\hline
	\bottomrule
	\end{tabular*}
	\label{tab:fitvals-ferro}
	\end{center}
\end{table}
\endgroup


\begin{figure}
	\centering
	\includegraphics[width=1.0\textwidth]{./fig/ferrimagnet_t_comp.png}
	\caption{Coercive field strength of a FeGd layer system over temperature}
	\label{fig:compensation-temperature}
\end{figure}
