%!TEX root = ../main.tex

\section{Ferromagnets: Pd/Co/Pd sample}
\label{sec:ferromagnet}

The first sample consists of two thin layers of Palladium (paramagnetic) and one
layer of Cobalt (ferromagnetic) of variable thickness $d$. The crystal lattice of the
trilayer system in total acts as a ferromagnet. The different magnetic properties of 
the sublattice however can cause strong changes in both magnitude and direction of
magnetization in special cases. These effects will in part be analysed in this 
section.

Before the measurement data is evaluated the applied model is presented. The 
Kerr-angle that is measured as a function of external magnetic field is proportional
to intrinsic magnetisation, we are effectively measuring a magnetic hysteresis modulo
some numerical constants. Expanding on the work in \cite{article} we can therefore 
impose a model to analytically describe the hysteresis curve traced out by the
Kerr-angle $\Upphi$ in the experiment. This model is presented in 
\autoref{eq:hysteresis-model}.

\begin{equation}
\label{eq:hysteresis-model}
\Upphi(H) = \Upphi_s \tanh\left( B ( H \pm H_c ) \right) + \mu_0 H.
\end{equation}

In the above \autoref{eq:hysteresis-model} there are several parameters that 
influence the shape of the hysteresis curve. These are:

\begin{itemize}
\item $\Upphi_s$ is the Kerr-angle at the saturation point. 
\item $B$\footnote{for lack of more fitting nomenclature in the literature} 
determines the magnetic hardness of the sample.
\item $\pm H_c$ is the coercive field strength, on the up/downsweep of $H$.
\end{itemize}


