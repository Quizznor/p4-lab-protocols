%!TEX root = ../main.tex

\section{Antiferromagnets: Co/Pd/Co sample}
\label{sec:antiferromagnet}



The second sample is a layered system analogous to the first. It consists of two thin cobalt layers (ferromagnetic) as well as a variable palladium layer (paramagnetic) with thickness $d$. 
When the sample was examined, however, no hysteresis curves could be measured, regardless of the palladium layer thickness. 
The impurities on the second sample were probably too large despite various approaches.

It was to be expected that at low layer thickness $d$ no hysteresis curve is measured and therefore no magnetization perpendicular to the surface. 
This is due to the fact that the magnetization of the thin Co layer still couples to the magnetization of the thick Co layer and is thus forced in a direction parallel to the surface. 
With increasing Pd layer thickness $d$, this coupling becomes weaker and the surface effects of the thin Co layer dominate, resulting in an effective magnetization perpendicular to the surface.
