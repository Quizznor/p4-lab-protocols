%!TEX root = ../main.tex

\section{Messaufbau}
\label{sec:setup}

Beide Messproben sind eingelagert in einen Tank der mithilfe von flüssigem Stickstoff
auf tiefe Temperaturen ($<\SI{150}{\kelvin}$) gekühlt werden kann. Zum anschließenden
aufhitzen der Halbleiter befindet sich ein Heizelement innerhalb des Tanks, dass 
extern über ein Tastenfeld bedient werden kann. Dieses Bedienfeld zeigt auch die 
aktuelle Temperatur des Heizelements an. Da diese nicht unbedingt der Temperatur der
Halbleiter entsprechen muss und zudem während der Messungen im Zehntel-Kelvin-Bereich
schwankt wird eine konservative Ungenauigkeit von $\pm\SI{1.5}{\kelvin}$ in der 
Temperatur der Proben geschätzt. Die Leit- und Hallspannung beider Proben wird
mittels vierer Golddrähte wie in \autoref{fig:germanium} gezeigt abgenommen. Die 
Messgenauigkeit des verfügbaren Voltmeters wird als unendlich angesehen. Damit sind 
Ungenauigkeiten der Spannugswerte lediglich der Ablesegenauigkeit geschuldet. Diese
beträgt für die Germanium-Probe $\Delta U_\text{Leit} = \SI{0.5}{\milli\volt}$ bzw.
$\Delta U_\text{Hall} = \SI{0.5}{\micro\volt}$. Die Ungenauigkeiten für den 
Galliumarsenid Halbleiter ist gegeben durch $\Delta U=\SI{5}{\micro\volt}$. Weitere
Informationen können dem Laborhandbuch \cite{Manual} entnommen werden.
