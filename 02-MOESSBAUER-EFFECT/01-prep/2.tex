%!TEX root = ../main.tex

\section{Mössbauer spectroscopy}
\label{sec:mössbauer-spectroscopy}

To measure the the natural linewidth of $^{57}$Fe with roughly \SI{5}{\nano\electronvolt}\footnote{\cite{Sch17}, Page 160} common measurement methods will fail. Even with high-resolution interferometers, some orders of magnitude are still missing to resolve such lines by direct measurement of the spectrum. 
Therefore we take the Mössbauer effect into consideration. It is possible to measure the resonant absorption by the transition rate of the photons. 
The detector does not need to be energy sensitive, it is sufficient to detect the quanta in this case. The magnitude of the transition displays the overlap of the probabilities for emission and absorption (see \autoref{fig:resonant-absorption}) of the natural lines. 
By varying the distance of the emission and absorption peaks we can make a statement about the line profile. \\
\\
If the gamma radiation source is in motion a \textbf{Doppler shift} occurs and the photon energy changes. Therefor we can slightly modify the photon energy by changing the velocity of the source (see  \autoref{eq:energy-doppler}). 

\begin{equation}
    \Delta E_{\gamma} = \pm \frac{v}{c} \cdot E
    \label{eq:energy-doppler}
\end{equation}
\\
With the measured transmission spectrum as a function of the velocity of the source one can make qualitative as well as quantitative statements about the element. 
Furthermore it is possible to observe three types of nuclear interaction: isometric shift, quadrupole splitting and hyperfine magnetic splitting. 
The FWHM of the transmission curve corresponds to the double natural linewidth of the transition.