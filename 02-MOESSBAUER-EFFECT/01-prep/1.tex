%!TEX root = ../main.tex

\section{Mössbauer effect}
\label{sec:mössbauer-effect}

The process of \textbf{resonant absorption} in nuclear physics describes the 
phenomenon of subsequent de- and excitation of two equal atoms to the same energy 
levels via one $\gamma$-quant. Consider for example an excited state of $^{57}$Fe,
that emits a photon with energy (roughly) \SI{14.4}{\kilo\electronvolt} during its 
transition to the ground state.

\begin{equation*}
^{57}\text{Fe}^*\;\longrightarrow\;^{57}\text{Fe}\;+\;\gamma  
\end{equation*}

In principle, one could use this emitted photon to excite another $^{57}$Fe atom to 
the higher energy state. The photon is absorbed resonantly by the atom during this 
process.

In reality, resonant absorption such as the Na-D-line only occurs under certain
circumstances. Due to conservation laws the energy $E_\gamma$ of the emitted photon
does not exactly equal the transition energy $E_0$, but is instead shifted downward
by the nuclear recoil energy. A similar analysis finds that the energy for absorption
of the same atom is shifted upwards. 

\begin{equation}
\underbrace{E_\gamma = E_0 - \frac{p_\gamma^2}{2m}}_\text{Emission} \qquad\qquad\qquad \underbrace{E_\gamma = E_0 + \frac{p_\gamma^2}{2m}}_\text{Absorption} 
\end{equation}

With the photon impulse $p_\gamma$ and atom mass $m$. If the line width introduced by
natural broadening, Doppler broadening or other  effects does not exceed the gap 
$\frac{p_\gamma^2}{m}$ resonant absorption cannot occur. This is visualised in 
\autoref{fig:resonant-absorption}.
