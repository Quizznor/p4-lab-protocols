%!TEX root = ../main.tex

\section{Conclusion}
\label{sec:conclusion}

The physics of the Mössbauer effect were tested experimentally. Different
characteristics of the various absorbers - such as their isometric shift - were 
quantified. Where possible, the obtained values are compared to values given by
literature. Overall, the analysis largely conforms to established theory. However,
large relative errors are found for all the listed measurement parameters. These are 
largely caused by statistical uncertainties of the Mössbauer spectrum fit parameters.
In future lab reports these could be diminished by using more powerful fitters than 
were used in this analysis ($\text{numpy},\;\text{scipy}$). Another promising idea is
a combined analysis of the whole Mössbauer spectrum, rather than fitting individual 
to absorption peaks. In theory, the fit parameters $\Upphi_0$ or the normalisation 
parameter $A$ should be the same for the different peaks. Using this knowledge, one
could construct a fit function that optimises these values globally and then fits 
the location and FWHM of the absorption peaks. This might result in a more stable
fit.
