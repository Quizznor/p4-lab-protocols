%!TEX root = ../main.tex

\section{Measuring the Mössbauer spectrum}
\label{sec:mössbauer-spectrum}

In the following section the Mössbauer spectrum of various materials is measured. The
process of how data is collected and analysed is presented using the example of an 
iron absorber. Other materials are analysed in a similar fashion.

\subsection{Iron}
\label{ssec:iron}

The iron absorber is placed in the beam path. The transmitted photons are counted for
roughly \SI{20}{\minute}. In this measuring mode, the MCS channel number corresponds
to a velocity $v$ at which the $\gamma$-source is moving relative to the iron target.
The 1024 channels available for readout are split into two 512-channel intervals (in 
the following adressed as Ch1 and Ch2) that differ in the acceleration of the source.It is assumed that the measurements of Ch1 and Ch2 respectively are uncorrelated.
As lined out in the lab manual \cite{Sch17} it is assumed that the maximum velocity 
$v_\text{max}=\SI{\pm10}{\milli\meter\per\second}$ corresponds to the edges of the 
channel intervals (i.e. Channel \#0 and \#1023 for positive velocity, channel \#511, 
and \#512 for negative velocity). Using this information, a relation between channel 
number $\mathcal{C}$ and $\gamma$-source velocity can be constructed as follows:

\begin{equation}
\label{eq:channel-to-velocity}
v(\mathcal{C}) = \SI{10}{\milli\meter\per\second}\cdot\left(\frac{\mathcal{C}-256}{256}\right).
\end{equation}

Because of this relative velocity the photons that are emitted at an energy of 
\SI{14.4}{\kilo\electronvolt} are Doppler shifted to slightly lower/higher energies.
If now a nuclear transition from state $|i\rangle$ to state $|f\rangle$ exists for an
iron atom in the crytal lattice where $E_f-E_i=E'$, there is a nonzero probability 
that the atom absorbs the photon and transitions to the higher energy state. 
Consequently, a dip in the photon spectrum at that specific energy (and by extension 
a specific velocity) can be observed. The resonance around this energy $E'$ can be 
modelled by a slightly modified Breit-Wigner shape presented in \autoref{eq:fitfunc}.

\begin{equation}
\label{eq:fitfunc}
N(v) = \Upphi_0 - \frac{A}{(v-v_0)^2 - (\frac{1}{2}\,\Gamma)^2},
\end{equation}

where $\Upphi_0$ is the integrated $\gamma$-flux (i.e. number of photons with energy 
$E\approx\SI{14.4}{\kilo\electronvolt}$). The normalisation factor $A$, $v_0$ the 
velocity of the $\gamma$-source at which the photons are shifted by the transition 
energy $E'$ and $\Gamma$ the full-width-at-half-maximum (FWHM) value of the 
absorption peak. Technically, using this parametrisation of the Breit-Wigner curve, 
$\Gamma$ should also appear in the numerator. To ensure a more stable fit result, 
it has however been absorbed in $A$.

Fitting \autoref{eq:fitfunc} to measurement data, six absorption peaks can be 
identified (see \autoref{fig:iron}) depending on the velocity of the $\gamma$-source.
The individual parameters optimised to model the observed spectrum are listed in 
\autoref{tab:iron}. It is important to note that absorption peak \#4 and \#5 are not 
properly fitted for data from Channel 2. The information gathered from these peaks
will be ignored in the following analysis. For every other absorption peak the
corresponding fit parameters $P$ will be combined to a mean value $\bar{P}$.

\begin{align*}
	\bar{P} &= \;\;P_\text{Ch1} + P_\text{Ch2} \\[0.5cm]
	\Delta\bar{P} &= \sqrt{\Delta P_\text{Ch1}^2 + \Delta P_\text{Ch2}^2  } 
\end{align*}

\begin{figure}
	\label{fig:iron}
	\includegraphics[width=1.0\textwidth]{./fig/Iron.png}
	\caption{Mössbauer spectrum of iron}{Six absorption peaks can be seen in the
	Mössbauer spectrum of ironSix absorption peaks can be seen in the
	Mössbauer spectrum of iron. They are labelled \#1 to \#6. The two measurement
	channels are colour-coded respectively.}
\end{figure}

\begingroup
\renewcommand{\arraystretch}{1.3}
\begin{table}
	\begin{center}
	\caption{Mössbauer spectrum fit parameters for iron}
	\begin{tabular*}{0.9\textwidth}{@{\extracolsep{\fill}} c|ccccc}
  \toprule
	\hline
  Peak \# & $\Upphi_0$ & $A$ & $v_0$ & $\Gamma$ & Channel \\
	\hline
  \multirow{2}{*}{\#1} & $10177\pm16$ & $12\pm3.9$ & $-4.02\pm0.02$ & $0.184\pm0.043$ & Ch1 \\
  			               & $10142\pm17$ & $21\pm6.0$ & $-4.11\pm0.02$ & $0.277\pm0.049$ & Ch2 \\
  			                 \hline
  \multirow{2}{*}{\#2} & $10187\pm53$ & $8\pm6.2$ & $-2.22\pm0.03$ & $0.161\pm0.104$ & Ch1 \\
  			               & $10108\pm38$ & $10\pm5.6$ & $-2.32\pm0.01$ & $0.184\pm0.080$ & Ch2 \\
  			                 \hline
  \multirow{2}{*}{\#3} & $10173\pm37$ & $8\pm5.2$ & $-0.49\pm0.02$ & $0.234\pm0.099$ & Ch1 \\
  			               & $10151\pm47$ & $9\pm5.6$ & $-0.51\pm0.03$ & $0.235\pm0.076$ & Ch2 \\
  			                 \hline
  \multirow{2}{*}{\#4} & $10092\pm50$ & $4\pm6.0$ & $0.82\pm0.02$ & $0.130\pm0.193$ & Ch1 \\
  			               & $10082\pm65$ & $2\pm6.1$ & $0.79\pm0.05$ & $0.000\pm\infty$ & Ch2 \\
  			                 \hline
  \multirow{2}{*}{\#5} & $10243\pm38$ & $15\pm5.4$ & $2.61\pm0.02$ & $0.266\pm0.048$ & Ch1 \\
  			               & $10075\pm24$ & $3\pm2.4$ & $2.51\pm0.02$ & $0.000\pm\infty$ & Ch2 \\
  			                 \hline
  \multirow{2}{*}{\#6} & $10173\pm17$ & $23\pm5.1$ & $4.30\pm0.01$ & $0.246\pm0.036$ & Ch1 \\
  			               & $10142\pm19$ & $21\pm6.9$ & $4.25\pm0.01$ & $0.250\pm0.058$ & Ch2 \\
  			                 \hline
    \bottomrule
		\end{tabular*}
		\label{tab:iron}
	\end{center}

\end{table}
\endgroup


\todo{Add isometric shift discussion}
