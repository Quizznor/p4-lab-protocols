%!TEX root = ../main.tex

\section{Energy calibration of the MCS analyser}
\label{sec:ecal}

As described in \autoref{sec:mössbauer-spectroscopy}, Mössbauer spectroscopy does not
rely on an extremely precise measurement of the photon energy. Due to resonant
absorption it is sufficient to merely count the number of photons transmitted through
an absorber to gather information about its atomic energy levels and existing nuclear
transitions. However, in order to allow for more detailed analysis in the following
evaluation the relative difference of transition energies needs to be quantified.
For this purpose an energy calibration is performed by fitting a modified Lorentzian
curve $f(\mathcal{C})$ given in \autoref{eq:ecal-fit} to the $\gamma$-spectrum of
$^{57}$Co. The results of this analysis are also given below. The found best fit as
well as the gathered $\gamma$-spectrum are shown in \autoref{fig:ecal}

\begin{equation}
\label{eq:ecal-fit}
f(\mathcal{C}) = \frac{A}{(\mathcal{C}^2-\omega_0^2)^2 + \gamma^2\omega_0^2}
\end{equation}

\begin{align*}
	A &= \SI{8.68\pm 0.36e+10}{} \\
	\omega_0 &= \SI{67.84\pm 0.08}{} \\
	\gamma &= \SI{23.25\pm 0.52}{}
\end{align*}

\begin{equation*}
\text{COV}(A,\omega_0,\gamma) =
\left[
\begin{array}{c c c}
	1.28\times10^{19} & 1.00\times10^{7} & 1.86\times10^{9} \\
	1.00\times10^{7} & 5.66\times10^{-3} & -6.41\times10^{-4} \\
	1.86\times10^{9} & -6.41\times10^{-4} & 2.75\times10^{-1}
\end{array}
\right]
\end{equation*}

\begin{equation*}
	\Delta f(\mathcal{C}) = \sqrt{ (\nabla f(\mathcal{C}))^T\;\text{COV}(A,\omega_0,\gamma)\;\nabla f(\mathcal{C})}
\end{equation*}

\begin{figure}
	\centering
	\includegraphics[width=1.0\textwidth]{./fig/calibration.png}
	\caption{The error on the measured counts is assumed to be Poissonian.}
	\label{fig:ecal}
\end{figure}

--
