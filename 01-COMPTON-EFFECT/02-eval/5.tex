%!TEX root = ../main.tex

\section{Z-dependency of differential cross section}
\label{sec:z-dependence}

Lastly, the differential cross section $\frac{\d\sigma}{\d\Omega}$ at one specific
angle ($\theta = \SI{20}{\degree}$) is compared for different elements. Assuming
\autoref{eq:klein-nishina} holds for a photon scattering off a single electron, the
cross section of the photon scattering off any electron inside an atom should scale
like

\begin{equation}
\label{eq:z-dependence}
\left( \frac{\d\sigma}{\d\Omega} \right)_\text{Atom} = Z\cdot\left( \frac{\d\sigma}{\d\Omega} \right)_\text{electron}
\end{equation}

This espically implies that the quantity $\frac{R\cdot A}{\rho\cdot Z}$ connecting
the measured event rate $R$ with the material specific density $\rho$, atomic number
$Z$, and atomic mass number $A$ should be a constant! This is tested by placing
targets of the same geometry, but different materials into the beam path. The number
of events seen by the scintillator detector is counted for $\SI{300}{\second}$. The
results as well as reference data is listed in \autoref{tab:z-dependence}. As can be
seen in the table, the quantity $\frac{R\cdot A}{\rho\cdot Z}$ is not constant. This
cannot be attributed to a faulty energy calibration, as all systematic errors remain
the same for these measurements and should cancel out upon comparison. The reason for
the nonconformity is unclear and may be caused by untreated detector systematics.

The measured rate for the lead target is a clear outlier at a mean deviation of
$~30\%$ relative to other measurements. This is most likely caused by the poor
material quality of the lead target. Due to previous handling the top part of the
lead target broke off. It therefore had a smaller geometry than all other targets,
the measured rate is expected to drop accordingly.

\begin{table}
	\begin{center}
	\caption{Z dependance of $R$}
	\begin{tabular*}{0.7\textwidth}{@{\extracolsep{\fill}} c|rrr}
		\toprule
    \textbf{Element} & \boldmath{$Z/A$} & \boldmath{$R$} & \boldmath{$\frac{R\cdot A}{\rho Z}$} \\
		\midrule
    Al & 13/26.982 & 49.00$\pm$0.95 & 37.66$\pm$0.73 \\
    Fe & 26/55.845 & 155.23$\pm$1.12 & 42.34$\pm$0.31 \\
    Cu & 29/63.546 & 166.09$\pm$1.14 & 40.62$\pm$0.28 \\
    Pb & 82/207.200 & 132.46$\pm$1.09 & 29.51$\pm$0.24 \\
		\bottomrule
		\end{tabular*}
	\end{center}
	\label{tab:energy-shift}
\end{table}

