\documentclass{00-include/protokollclass}
% Main File - Based on protokollclass.cls
% Comments are mostly in English (and some in German, concerning the Praktikum)
% ------------------------------------------------------------------------------
% Further files in folder:
%  - include/cmds.tex (for macros and additional commands)
%  - include/kitlogo.pdf (for titlepage)
%  - lit.bib (bibtex bibliography database)
%  - include/titlepage.tex (for layout of titelpage)
% ------------------------------------------------------------------------------
% Useful Supplied Packages:
% amsmath, amssymb, mathtools, bbm, upgreek, nicefrac,
% siunitx, varioref, booktabs, graphicx, tikz, multicol





%% ---------------------------------------------
%% |    Informationen über dieses Protokoll    |
%% ---------------------------------------------
\newcommand{\praktikum}{P4}                % P1 oder P2
\newcommand{\semester}{WS20/21}            % z.B. "WS14/15" oder "SS15"

\newcommand{\wochentag}{Mo}                % Mo, Di, Mi oder Do
\newcommand{\gruppennr}{22}                % Zweistellige Gruppennummer

\newcommand{\nachnamea}{Filip}             % Nachname des ersten Praktikanten
\newcommand{\vornamea}{Paul}               % Vorname des ersten Praktikanten
\newcommand{\nachnameb}{Janic}             % Nachname des zweiten Praktikanten
\newcommand{\vornameb}{Beck}               % Vorname des zweiten Praktikanten

\newcommand{\emailadressen}{}
% optionale Angabe von Emailadresse(n) für den Kontakt mit dem Betreuer

\newcommand{\versuch}{Comptoneffekt}       % Name des Versuchs
\newcommand{\versuchsnr}{}               % bitte die korrekte Nummer dem
                                           % Arbeitsplatz am Versuchstag
                                           % entnehmen
\newcommand{\fehlerrechnung}{Ja}           % Ob Fehlerrechnung im Versuch
                                           % durchgeführt wurde oder nicht

\newcommand{\betreuer}{Roxanne Turcotte}   % Name des zuständigen Betreuers
\newcommand{\durchgefuehrt}{16.11.20}      % Datum, an dem der Versuch
                                           % durchgeführt wurde

\newcommand{\protokollseiten}{2}           % Seitenlänge des Messprotokolls
\newcommand{\MaxKapitel}{10}





%% --------------------------------------
%% |    Settings for Word Separation    |
%% --------------------------------------
% Help for separation:
% In German package the following hints are additionally available:
% "- = Additional separation
% "| = Suppress ligation and possible separation (e.g. Schaf"|fell)
% "~ = Hyphenation without separation (e.g. bergauf und "~ab)
% "= = Hyphenation with separation before and after
% "" = Separation without a hyphenation (e.g. und/""oder)

% Describe separation hints here:
\hyphenation
{
    über-nom-me-nen an-ge-ge-be-nen
    %Pro-to-koll-in-stan-zen
    %Ma-na-ge-ment  Netz-werk-ele-men-ten
    %Netz-werk Netz-werk-re-ser-vie-rung
    %Netz-werk-adap-ter Fein-ju-stier-ung
    %Da-ten-strom-spe-zi-fi-ka-tion Pa-ket-rumpf
    %Kon-troll-in-stanz
}

% um die Titelseite per PDF-reader auszufüllen. Vorgefertigte Daten
% können in Datei 'data.tex' modifiziert werden.
%\setboolean{forminput}{true}
% um die Anmerkungen zu den Textfeldern anzeigen zu lassen
%\setboolean{showannotations}{true}
% Erneuern der Seitenzahl in jedem Kapitel
%\setboolean{chapResetPageNumb}{true}
% Einbinden der Kapitelnummer in der Seitenzahl
%\setboolean{chapWiseNumb}{true}
% english or ngerman (new german für neue deutsche Rechtschreibung statt german)
\SelectLanguage{english}





%% -----------------------
%% |    Main Document    |
%% -----------------------
\begin{document}
    % Titlepage und ToC
    \FrontMatter

    % coordinates for background border
\newcommand{\diameter}{20}
\newcommand{\xone}{-15}
\newcommand{\xtwo}{160}
\newcommand{\yone}{15}
\newcommand{\ytwo}{-253}

\newcommand{\hoehea}{60}
\newcommand{\hoeheb}{60}




\begin{titlepage}
    % background border
    \begin{tikzpicture}[overlay]
    \draw[color=gray]
            (\xone mm, \yone mm)
      -- (\xtwo mm, \yone mm)
    arc (90:0:\diameter pt)
      -- (\xtwo mm + \diameter pt , \ytwo mm)
        -- (\xone mm + \diameter pt , \ytwo mm)
    arc (270:180:\diameter pt)
        -- (\xone mm, \yone mm);
    \end{tikzpicture}

    % KIT logo
    \begin{textblock}{10}[0,0](4.5,2.5)
        \includegraphics[width=.25\textwidth]{00-include/kitlogo.pdf}
    \end{textblock}
    \changefont{phv}{m}{n}    % helvetica
    \begin{textblock}{10}[0,0](5.5,2.2)
        \begin{flushright}
            \Large FAKULTÄT FÜR PHYSIK\\Praktikum Moderne Physik
        \end{flushright}
    \end{textblock}

    \begin{textblock}{10}[0,0](4.2,3.1)
        \begin{tikzpicture}[overlay]
        \draw[color=gray]
            (\xone mm + 5 mm, -12 mm)
         -- (\xtwo mm + \diameter pt - 5 mm, -12 mm);
        \end{tikzpicture}
    \end{textblock}

    \Large
    % Zeile 1
    \begin{textblock}{12}[0,0](3.58,4.4)
        \mytextfield{Praktikum}{\praktikum}{0.9cm}{17pt}{}{}{Praktikum}
    \end{textblock}
    \begin{textblock}{12}[0,0](6.58,4.4)
       \mytextfield{Gruppe}{\gruppennr}{1.06cm}{17pt}
                   {\#\#}{2}{Gruppennummer}
    \end{textblock}

    \begin{textblock}{12}[0,0](10.2,4.1)
    \fbox{\textcolor{black}{\rule{15pt}{15pt}}}{\;\huge\textbf{Mo}}
    \fbox{\textcolor{white}{\rule{15pt}{15pt}}}{\;\huge\textbf{Mi}}
    \end{textblock}

    \begin{textblock}{12}[0,0](10.34,4.6)
        \scriptsize Zutreffendes bitte ausfüllen
    \end{textblock}

    \begin{textblock}{12}[0,0](14,4.16)
        \huge\textbf{\semester}
    \end{textblock}

    % Zeile 2
    \begin{textblock}{12}[0,0](3.58,5.3)
          \mytextfield{Namen}{\vornamea\;\nachnamea\qquad\emailadressea}{14cm}{17pt}{}{0}{Name 1}
    \end{textblock}

    % Zeile 3
    \begin{textblock}{12}[0,0](3.58,5.9)
          \mytextfield{Namen}{\vornameb\;\nachnameb\qquad\qquad\emailadresseb}{14cm}{17pt}{}{0}{Name 2}
    \end{textblock}

    % Zeile 4
    \begin{textblock}{12}[0,0](3.58,7)
        \mytextfield{Versuch}{\versuch}{13.8cm}{14pt}{}{0}{Versuch}
    \end{textblock}

    % Zeile 6
    \begin{textblock}{12}[0,0](3.58,7.55)
        \mytextfield{Betreuer}{\betreuer}{7cm}{17pt}{}{0}{Betreuer}
    \end{textblock}
    \begin{textblock}{12}[0,0](10.82,7.55)
        \mytextfield{Durchgeführt am}{\durchgefuehrt}{2.53cm}{17pt}{}{8}{Durchfuehrung}
    \end{textblock}

    % Querstrich
    \begin{textblock}{20}[0,0](0,7.9)\tiny\centering
        Wird vom Betreuer ausgefüllt.
    \end{textblock}
    \begin{tikzpicture}[overlay]
    \draw[color=gray]
        (\xone mm + 5 mm, -95 mm)
     -- (\xtwo mm + \diameter pt - 5 mm, -95 mm);
    \end{tikzpicture}

    % Zeile 1
    \begin{textblock}{12}[0,0](3.65,8.57)
        \myTtextfield{1. Abgabe am}{}{2.5cm}{17pt}
                     {}
    \end{textblock}

    % Block 1
    \begin{tikzpicture}[overlay]
    \draw[color=gray]
        (\xone mm + 10 mm, -107.5 mm)
     -- (\xtwo mm + \diameter pt - 10 mm, -107.5 mm)
     -- (\xtwo mm + \diameter pt - 10 mm, -107.5 mm - \hoehea mm)
     -- (\xone mm + 10 mm, -107.5 mm - \hoehea mm)
     -- (\xone mm + 10 mm, -107.5 mm);
    \end{tikzpicture}
    \begin{textblock}{20}[0,0](3.8,9.2)
        \myTtextfield{Rückgabe am}{}{2.5cm}{17pt}{}
    \end{textblock}
    \begin{textblock}{20}[0,0](8.7,9.2)
        \smash{Kommentar:}
    \end{textblock}

    % Zeile 2
    \begin{textblock}{12}[0,0](3.65,12.6)
        \myTtextfield{2. Abgabe am}{}{2.5cm}{17pt}{}
    \end{textblock}

    % Block 2
    \begin{tikzpicture}[overlay]
    \draw[color=gray]
        (\xone mm + 10 mm, -180 mm)
     -- (\xtwo mm + \diameter pt - 10 mm, -180 mm)
     -- (\xtwo mm + \diameter pt - 10 mm, -180 mm - \hoehea mm)
     -- (\xone mm + 10 mm, -180 mm - \hoehea mm)
     -- (\xone mm + 10 mm, -180 mm);
    \end{tikzpicture}
    \begin{textblock}{12}[0,0](4,13.25)
        \smash{Ergebnis:~~~~+~~~/~~~0~~~/~~~-}
    \end{textblock}
    \begin{textblock}{12}[0,0](3.8,13.72)
        \myTtextfield{Datum}{}{2.5cm}{17pt}{}
    \end{textblock}
    \begin{textblock}{12}[0,0](8.3,13.25)
        \myTtextfield{Handzeichen}{}{5.5cm}{17pt}{}
    \end{textblock}
    \begin{textblock}{12}[0,0](7.6,13.72)\Large
        \smash{Kommentar:}
    \end{textblock}

    % lowest text blocks concerning the KIT
    \begin{textblock}{10}[0,0](4,16.8)
        \tiny{KIT -- Universität des Landes Baden-Württemberg und nationales %
              Forschungszentrum in der Helmholtz-Gemeinschaft}
    \end{textblock}
    \begin{textblock}{10}[0,0](14,16.75)
        \large{\textbf{www.kit.edu}}
    \end{textblock}
\end{titlepage}
 %\cleardoublepage
    \tableofcontents

    % % Uncomment this if you want a list of figures and tables
    % \begingroup \let\clearpage\relax
    % \listoffigures
    % \listoftables
    % \endgroup
    % \cleardoublepage

    % Contents
    \MainMatter

    \pseudochapter[\protokollseiten]{Measurement report}

    % Uncomment this line to include the measurement report
    % \includepdf[pages=-]{data/measurement_report.pdf}

    \chapter{Theory \& Preparation}

    \foreach \c in {1,2,...,\MaxKapitel}{%
      \IfFileExists{01-prep/\c}{%
        \input{01-prep/\c}{}
      }%
    }%

    \chapter{Experiment \& Evaluation}

    \foreach \c in {1,2,...,\MaxKapitel}{%
      \IfFileExists{02-eval/\c}{%
        \input{02-eval/\c}{}
      }%
    }%

      % % appendix for more or less interesting calculations
    % \Appendix
    % \chapter*{\appendixname} \addcontentsline{toc}{chapter}{\appendixname}
    % % to make the appendix appear in ToC without number. \appendixname =
    % % Appendix or Anhang (depending on chosen language)
    % \input{./chap/appendix.tex} %\cleardoublepage



    % Bibliography
    \TheBibliography

    % BIBTEX
    % use if you want citations to appear even if they are not referenced to:
    % \nocite{*} or maybe \nocite{Kon64,And59} for specific entries
    %\nocite{*}
    \bibliographystyle{babalpha}
    \bibliography{lit.bib}

    % THEBIBLIOGRAPHY
    %\begin{thebibliography}{000}
    %    \bibitem{ident}Entry into Bibliography.
    %\end{thebibliography}
\end{document}
