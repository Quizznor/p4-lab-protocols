%!TEX root = ../main.tex

\section{Compton spectrum}
\label{sec:compton-spectrum}

Without knowing anything about the distribution of energies or deflection angles of 
the scattered electrons, by examining \autoref{eq:electron-energy} it can already be 
established that the energy the electron gains from the interaction is directly
proportional to $\left(1-\cos\theta\right)$, or in other words, by how much the 
photon is scattered away from its original path. It follows that for $\theta\,=\,$
\SI{180}{\degree} the electron gains a maximal energy of 

\begin{equation}
\label{eq:maximal-energy}
E_\text{max}=\frac{E_\gamma}{1+\frac{m_e c^2}{E_\gamma}}.
\end{equation}

Since the photon physically cannot dump more energy by this process, a sharp drop in
the Compton spectrum at $E_\text{max}$ is expected. 


\todo{Discussion of angular distribution, compton spectrum}
