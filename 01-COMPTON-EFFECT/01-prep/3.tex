%!TEX root = ../main.tex

\section{Compton spectrum}
\label{sec:compton-spectrum}

Without knowing anything about the distribution of energies or deflection angles of
the scattered electrons, by examining \autoref{eq:electron-energy} it can already be
established that the energy the electron gains from the interaction is directly
proportional to $\left(1-\cos\theta\right)$, or in other words, by how much the
photon is scattered away from its original path. It follows that for \\
$\theta\,=\,$\SI{180}{\degree} the electron gains a maximum energy of

\begin{equation}
\label{eq:maximal-energy}
E_\text{max}=\frac{E_\gamma}{1+\frac{m_e c^2}{E_\gamma}}.
\end{equation}

Since the photon physically cannot dump more energy by this process, a sharp drop in
the Compton spectrum at $E_\text{max}$ is expected. This characteristic drop-off is
commonly called the \textbf{Compton edge}. Energetically lower than this cutoff lays
the \textbf{Compton continuum}, or main part of the spectrum. Over a wide range of
energies that correspond to scattering angles $\theta\in[\SI{0}{\degree},\SI{180}
{\degree}]$ the flux of electrons remains approximately constant. This follows
directly from \autoref{eq:klein-nishina-cross-section}.

\begin{figure}
	\includegraphics[width=1.0\textwidth]{fig/compton-spectrum}
	\caption{An idealised Compton spectrum. A relatively constant flux of
	electrons is measured up to an energy of $E_\text{max}$, where the spectrum
	sharply drops off. At the high-energy end of the spectrum a gaussian shaped
	photopeak is visible.}
	\label{fig:compton-spectrum}
\end{figure}

Lastly, an idealised Compton spectrum as depicted in \autoref{fig:compton-spectrum}
will also display a characteristic \textbf{photopeak}. This photopeak is caused by
photons directly interacting with detector material via the photoelectric effect.
In this case, the entire energy of the photon is dumped inside the detector. It is
therefore a helpful reference point for calibrations, albeit not being part of the
Compton spectrum itself.

The measured Compton spectrum discussed in \autoref{chap:experiment} will display
several other characteristics such as a prominent X-ray line or a backscatter peak.
These properties are dependant on the experimental setup. As such they will be
discussed in the appropriate sections of \autoref{chap:experiment}.
