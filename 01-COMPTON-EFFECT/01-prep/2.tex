%!TEX root = ../main.tex

\section{Cross section}
\label{sec:compton-cross-section}

Compton scattering is the dominating effect by which photons with an energy between
\SI{100}{\kilo\electronvolt} and \SI{10}{\mega\electronvolt} interact with matter
\cite{damashek1970forward}. A theoretical description of the processes cross section
is given by the \textbf{Klein-Nishina formul} (KN).

\begin{equation}
\label{eq:klein-nishina-cross-section}
\frac{\d\sigma}{\d\Omega}\,_\text{KN} = \frac{\alpha^2}{2m_e}\left(\frac{E_{\gamma,f}}{E_{\gamma,i}}\right)^2\left[\frac{E_{\gamma,f}}{E_{\gamma,i}}+\frac{E_{\gamma,i}}{E_{\gamma,f}}-\sin^2\theta\right]
\end{equation}

Integrating over all solid angles and defining $x=\frac{E_{\gamma,i}}{m_e}$, one
obtains the total cross section.

\begin{equation}
\label{eq:total-cross-section}
\sigma_\text{tot.} = \int \frac{\d\sigma}{\d\Omega}\,\d\Omega = \frac{\pi\alpha^2}{m_e^2}\frac{1}{x^3}\left(\frac{2x(2+x(1+x)(8+x))}{(1+2x)^2} + ((x-2)x-2\log(1+2x)\right)
\end{equation}

In the low-energy limit of $x\ll1$ \autoref{eq:total-cross-section} simplifies to the
\textbf{Thomson cross section}, whereas in the high-energy limit $x\rightarrow\infty$
we expand in $x$ to find the following.

\begin{align*}
x\;\:\ll\;\;1&:\qquad\sigma_\text{tot.} = \frac{8\pi\alpha^2}{m_e^2} \\
x\longrightarrow\infty&:\qquad\sigma_\text{tot.} = \frac{\pi\alpha^2}{x m_e^2}\left(\frac{1}{2} + \log2x\right) 
\end{align*}

As it turns out, the cross section is constant for low-energy photons, where 
ionisation is more probable. Furthermore the likeliness of an interaction actually 
decreases for high-energy photons and eventually drops to zero. This behaviour can 
also be seen in \autoref{fig:compton-cross-section} and explains the importance of 
Compton scattering over the intermediate ranges of energy from \SI{100}{\kilo
\electronvolt} to \SI{10}{\mega\electronvolt}.

