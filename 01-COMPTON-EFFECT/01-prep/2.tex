%!TEX root = ../main.tex

\section{Cross section}
\label{sec:compton-cross-section}

Compton scattering is the dominating effect by which photons with an energy between
\SI{100}{\kilo\electronvolt} and \SI{10}{\mega\electronvolt} interact with matter
\cite{damashek1970forward}. A theoretical description of the processes cross section
is given by the \textbf{Klein-Nishina formula} (KN).

\begin{equation}
\label{eq:klein-nishina-cross-section}
\frac{\d\sigma}{\d\Omega}\,_\text{KN} = \frac{\alpha^2}{2m_e}\left(\frac{E_{\gamma,f}}{E_{\gamma,i}}\right)^2\left[\frac{E_{\gamma,f}}{E_{\gamma,i}}+\frac{E_{\gamma,i}}{E_{\gamma,f}}-\sin^2\theta\right]
\end{equation}

Integrating over all solid angles and defining $x=\frac{E_{\gamma,i}}{m_e}$, one
obtains the total cross section.

\begin{equation}
\label{eq:total-cross-section}
\sigma_\text{tot.} = \int \frac{\d\sigma}{\d\Omega}\,\d\Omega = \frac{\pi\alpha^2}{m_e^2}\frac{1}{x^3}\left(\frac{2x(2+x(1+x)(8+x))}{(1+2x)^2} + ((x-2)x-2\log(1+2x)\right)
\end{equation}

In the low-energy limit of $x\ll1$ \autoref{eq:total-cross-section} simplifies to
a constant called the \textbf{Thomson cross section}, whereas in the high-energy
limit $x\rightarrow\infty$ we expand in $x$ to find that the cross section vanishes.
This behaviour can also be seen in \autoref{fig:compton-cross-section}.

\begin{align*}
x\;\:\ll\;\;1&:\qquad\sigma_\text{tot.} = \frac{8\pi\alpha^2}{m_e^2} \approx \SI{0.6652}{\barn}\\
x\longrightarrow\infty&:\qquad\sigma_\text{tot.} = \frac{\pi\alpha^2}{x m_e^2}\left(\frac{1}{2} + \log2x\right)
\end{align*}

Naively, one would therefore expect Compton scattering to be an important process
for low to intermediate energy ranges of the electromagnetic spectrum, where the
cross section $\sigma_\text{Tot.}$ is non-negligible. This is however not the case.
Various other processes such as the photoelectric effect or Rayleigh scattering
dominate the low energy regime and render Compton scattering only a fringe case.
With higher photon energies, the cross section for the aforementioned processes
drop off, and increase the importance of Compton scattering for gammas carrying
an energy of \SI{100}{\kilo\electronvolt} up to \SI{10}{\mega\electronvolt} .
